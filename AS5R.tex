% Options for packages loaded elsewhere
\PassOptionsToPackage{unicode}{hyperref}
\PassOptionsToPackage{hyphens}{url}
%
\documentclass[
]{article}
\usepackage{amsmath,amssymb}
\usepackage{lmodern}
\usepackage{ifxetex,ifluatex}
\ifnum 0\ifxetex 1\fi\ifluatex 1\fi=0 % if pdftex
  \usepackage[T1]{fontenc}
  \usepackage[utf8]{inputenc}
  \usepackage{textcomp} % provide euro and other symbols
\else % if luatex or xetex
  \usepackage{unicode-math}
  \defaultfontfeatures{Scale=MatchLowercase}
  \defaultfontfeatures[\rmfamily]{Ligatures=TeX,Scale=1}
\fi
% Use upquote if available, for straight quotes in verbatim environments
\IfFileExists{upquote.sty}{\usepackage{upquote}}{}
\IfFileExists{microtype.sty}{% use microtype if available
  \usepackage[]{microtype}
  \UseMicrotypeSet[protrusion]{basicmath} % disable protrusion for tt fonts
}{}
\makeatletter
\@ifundefined{KOMAClassName}{% if non-KOMA class
  \IfFileExists{parskip.sty}{%
    \usepackage{parskip}
  }{% else
    \setlength{\parindent}{0pt}
    \setlength{\parskip}{6pt plus 2pt minus 1pt}}
}{% if KOMA class
  \KOMAoptions{parskip=half}}
\makeatother
\usepackage{xcolor}
\IfFileExists{xurl.sty}{\usepackage{xurl}}{} % add URL line breaks if available
\IfFileExists{bookmark.sty}{\usepackage{bookmark}}{\usepackage{hyperref}}
\hypersetup{
  pdftitle={AS5},
  pdfauthor={Bowen Zheng},
  hidelinks,
  pdfcreator={LaTeX via pandoc}}
\urlstyle{same} % disable monospaced font for URLs
\usepackage[margin=1in]{geometry}
\usepackage{color}
\usepackage{fancyvrb}
\newcommand{\VerbBar}{|}
\newcommand{\VERB}{\Verb[commandchars=\\\{\}]}
\DefineVerbatimEnvironment{Highlighting}{Verbatim}{commandchars=\\\{\}}
% Add ',fontsize=\small' for more characters per line
\usepackage{framed}
\definecolor{shadecolor}{RGB}{248,248,248}
\newenvironment{Shaded}{\begin{snugshade}}{\end{snugshade}}
\newcommand{\AlertTok}[1]{\textcolor[rgb]{0.94,0.16,0.16}{#1}}
\newcommand{\AnnotationTok}[1]{\textcolor[rgb]{0.56,0.35,0.01}{\textbf{\textit{#1}}}}
\newcommand{\AttributeTok}[1]{\textcolor[rgb]{0.77,0.63,0.00}{#1}}
\newcommand{\BaseNTok}[1]{\textcolor[rgb]{0.00,0.00,0.81}{#1}}
\newcommand{\BuiltInTok}[1]{#1}
\newcommand{\CharTok}[1]{\textcolor[rgb]{0.31,0.60,0.02}{#1}}
\newcommand{\CommentTok}[1]{\textcolor[rgb]{0.56,0.35,0.01}{\textit{#1}}}
\newcommand{\CommentVarTok}[1]{\textcolor[rgb]{0.56,0.35,0.01}{\textbf{\textit{#1}}}}
\newcommand{\ConstantTok}[1]{\textcolor[rgb]{0.00,0.00,0.00}{#1}}
\newcommand{\ControlFlowTok}[1]{\textcolor[rgb]{0.13,0.29,0.53}{\textbf{#1}}}
\newcommand{\DataTypeTok}[1]{\textcolor[rgb]{0.13,0.29,0.53}{#1}}
\newcommand{\DecValTok}[1]{\textcolor[rgb]{0.00,0.00,0.81}{#1}}
\newcommand{\DocumentationTok}[1]{\textcolor[rgb]{0.56,0.35,0.01}{\textbf{\textit{#1}}}}
\newcommand{\ErrorTok}[1]{\textcolor[rgb]{0.64,0.00,0.00}{\textbf{#1}}}
\newcommand{\ExtensionTok}[1]{#1}
\newcommand{\FloatTok}[1]{\textcolor[rgb]{0.00,0.00,0.81}{#1}}
\newcommand{\FunctionTok}[1]{\textcolor[rgb]{0.00,0.00,0.00}{#1}}
\newcommand{\ImportTok}[1]{#1}
\newcommand{\InformationTok}[1]{\textcolor[rgb]{0.56,0.35,0.01}{\textbf{\textit{#1}}}}
\newcommand{\KeywordTok}[1]{\textcolor[rgb]{0.13,0.29,0.53}{\textbf{#1}}}
\newcommand{\NormalTok}[1]{#1}
\newcommand{\OperatorTok}[1]{\textcolor[rgb]{0.81,0.36,0.00}{\textbf{#1}}}
\newcommand{\OtherTok}[1]{\textcolor[rgb]{0.56,0.35,0.01}{#1}}
\newcommand{\PreprocessorTok}[1]{\textcolor[rgb]{0.56,0.35,0.01}{\textit{#1}}}
\newcommand{\RegionMarkerTok}[1]{#1}
\newcommand{\SpecialCharTok}[1]{\textcolor[rgb]{0.00,0.00,0.00}{#1}}
\newcommand{\SpecialStringTok}[1]{\textcolor[rgb]{0.31,0.60,0.02}{#1}}
\newcommand{\StringTok}[1]{\textcolor[rgb]{0.31,0.60,0.02}{#1}}
\newcommand{\VariableTok}[1]{\textcolor[rgb]{0.00,0.00,0.00}{#1}}
\newcommand{\VerbatimStringTok}[1]{\textcolor[rgb]{0.31,0.60,0.02}{#1}}
\newcommand{\WarningTok}[1]{\textcolor[rgb]{0.56,0.35,0.01}{\textbf{\textit{#1}}}}
\usepackage{graphicx}
\makeatletter
\def\maxwidth{\ifdim\Gin@nat@width>\linewidth\linewidth\else\Gin@nat@width\fi}
\def\maxheight{\ifdim\Gin@nat@height>\textheight\textheight\else\Gin@nat@height\fi}
\makeatother
% Scale images if necessary, so that they will not overflow the page
% margins by default, and it is still possible to overwrite the defaults
% using explicit options in \includegraphics[width, height, ...]{}
\setkeys{Gin}{width=\maxwidth,height=\maxheight,keepaspectratio}
% Set default figure placement to htbp
\makeatletter
\def\fps@figure{htbp}
\makeatother
\setlength{\emergencystretch}{3em} % prevent overfull lines
\providecommand{\tightlist}{%
  \setlength{\itemsep}{0pt}\setlength{\parskip}{0pt}}
\setcounter{secnumdepth}{-\maxdimen} % remove section numbering
\ifluatex
  \usepackage{selnolig}  % disable illegal ligatures
\fi

\title{AS5}
\author{Bowen Zheng}
\date{9/30/2021}

\begin{document}
\maketitle

\hypertarget{r-markdown}{%
\subsection{R Markdown}\label{r-markdown}}

This is an R Markdown document. Markdown is a simple formatting syntax
for authoring HTML, PDF, and MS Word documents. For more details on
using R Markdown see \url{http://rmarkdown.rstudio.com}.

When you click the \textbf{Knit} button a document will be generated
that includes both content as well as the output of any embedded R code
chunks within the document. You can embed an R code chunk like this:

\begin{enumerate}
\def\labelenumi{(\alph{enumi})}
\tightlist
\item
  Use the rnorm() function to generate a predictor X of length n=100, as
  well as a noise vector of length n=100.
\end{enumerate}

\begin{Shaded}
\begin{Highlighting}[]
\FunctionTok{set.seed}\NormalTok{(}\DecValTok{1}\NormalTok{)}
\NormalTok{X }\OtherTok{\textless{}{-}} \FunctionTok{rnorm}\NormalTok{(}\DecValTok{100}\NormalTok{)}
\NormalTok{noise }\OtherTok{\textless{}{-}} \FunctionTok{rnorm}\NormalTok{(}\DecValTok{100}\NormalTok{)}
\end{Highlighting}
\end{Shaded}

\begin{enumerate}
\def\labelenumi{(\alph{enumi})}
\setcounter{enumi}{1}
\tightlist
\item
  Generate a response vector Y of length n=100 according to the model
\end{enumerate}

Y=β0+β1X+β2X2+β3X3+ϵ

Where β0, β1, β2, and β3are constants of your choice.

\begin{Shaded}
\begin{Highlighting}[]
\NormalTok{Y }\OtherTok{\textless{}{-}} \DecValTok{2} \SpecialCharTok{+} \DecValTok{1}\SpecialCharTok{*}\NormalTok{X }\SpecialCharTok{+} \DecValTok{3}\SpecialCharTok{*}\NormalTok{X}\SpecialCharTok{\^{}}\DecValTok{2} \SpecialCharTok{{-}} \FloatTok{0.1}\SpecialCharTok{*}\NormalTok{X}\SpecialCharTok{\^{}}\DecValTok{3} \SpecialCharTok{+}\NormalTok{ noise}
\end{Highlighting}
\end{Shaded}

\begin{enumerate}
\def\labelenumi{(\alph{enumi})}
\setcounter{enumi}{2}
\tightlist
\item
  Use the regsubsets() function to perform best subset selection in
  order to choose the best model containing the predictors
  X,X2,\ldots,X10. What is the best model obtained according to Cp, BIC,
  and adjusted R2? Show some plots to provide evidence for your answer,
  and report the coefficients of the best model obtained. Note you will
  need to use the data.frame() function to create a single data set
  containing both X and Y.
\end{enumerate}

\begin{Shaded}
\begin{Highlighting}[]
\FunctionTok{library}\NormalTok{(leaps)}
\NormalTok{data.full }\OtherTok{\textless{}{-}}  \FunctionTok{data.frame}\NormalTok{(}\StringTok{"y"} \OtherTok{=}\NormalTok{ Y, }\StringTok{"x"} \OtherTok{=}\NormalTok{ X)}
\NormalTok{mod.full }\OtherTok{\textless{}{-}}  \FunctionTok{regsubsets}\NormalTok{(y }\SpecialCharTok{\textasciitilde{}} \FunctionTok{poly}\NormalTok{(x, }\DecValTok{10}\NormalTok{, }\AttributeTok{raw=}\ConstantTok{TRUE}\NormalTok{), }\AttributeTok{data =}\NormalTok{ data.full, }\AttributeTok{nvmax =} \DecValTok{10}\NormalTok{)}
\NormalTok{mod.summary }\OtherTok{\textless{}{-}}  \FunctionTok{summary}\NormalTok{(mod.full)}
\FunctionTok{which.min}\NormalTok{(mod.summary}\SpecialCharTok{$}\NormalTok{cp)}
\end{Highlighting}
\end{Shaded}

\begin{verbatim}
## [1] 4
\end{verbatim}

\begin{Shaded}
\begin{Highlighting}[]
\FunctionTok{which.min}\NormalTok{(mod.summary}\SpecialCharTok{$}\NormalTok{bic)}
\end{Highlighting}
\end{Shaded}

\begin{verbatim}
## [1] 2
\end{verbatim}

\begin{Shaded}
\begin{Highlighting}[]
\FunctionTok{which.max}\NormalTok{(mod.summary}\SpecialCharTok{$}\NormalTok{adjr2)}
\end{Highlighting}
\end{Shaded}

\begin{verbatim}
## [1] 4
\end{verbatim}

\begin{Shaded}
\begin{Highlighting}[]
\FunctionTok{plot}\NormalTok{(mod.summary}\SpecialCharTok{$}\NormalTok{cp, }\AttributeTok{xlab=}\StringTok{"Subset Size"}\NormalTok{, }\AttributeTok{ylab=}\StringTok{"Cp"}\NormalTok{, }\AttributeTok{pch=}\DecValTok{20}\NormalTok{, }\AttributeTok{type=}\StringTok{"l"}\NormalTok{)}
\FunctionTok{points}\NormalTok{(}\DecValTok{3}\NormalTok{, mod.summary}\SpecialCharTok{$}\NormalTok{cp[}\DecValTok{3}\NormalTok{], }\AttributeTok{pch=}\DecValTok{4}\NormalTok{, }\AttributeTok{lwd=}\DecValTok{7}\NormalTok{)}
\end{Highlighting}
\end{Shaded}

\includegraphics{AS5R_files/figure-latex/unnamed-chunk-4-1.pdf}

\begin{Shaded}
\begin{Highlighting}[]
\FunctionTok{plot}\NormalTok{(mod.summary}\SpecialCharTok{$}\NormalTok{bic, }\AttributeTok{xlab=}\StringTok{"Subset Size"}\NormalTok{, }\AttributeTok{ylab=}\StringTok{"BIC"}\NormalTok{, }\AttributeTok{pch=}\DecValTok{20}\NormalTok{, }\AttributeTok{type=}\StringTok{"l"}\NormalTok{)}
\FunctionTok{points}\NormalTok{(}\DecValTok{3}\NormalTok{, mod.summary}\SpecialCharTok{$}\NormalTok{bic[}\DecValTok{3}\NormalTok{], }\AttributeTok{pch=}\DecValTok{4}\NormalTok{, }\AttributeTok{lwd=}\DecValTok{7}\NormalTok{)}
\end{Highlighting}
\end{Shaded}

\includegraphics{AS5R_files/figure-latex/unnamed-chunk-5-1.pdf}

\begin{Shaded}
\begin{Highlighting}[]
\FunctionTok{plot}\NormalTok{(mod.summary}\SpecialCharTok{$}\NormalTok{adjr2, }\AttributeTok{xlab=}\StringTok{"Subset Size"}\NormalTok{, }\AttributeTok{ylab=}\StringTok{"Adjusted R2"}\NormalTok{, }\AttributeTok{pch=}\DecValTok{20}\NormalTok{, }\AttributeTok{type=}\StringTok{"l"}\NormalTok{)}
\FunctionTok{points}\NormalTok{(}\DecValTok{3}\NormalTok{, mod.summary}\SpecialCharTok{$}\NormalTok{adjr2[}\DecValTok{3}\NormalTok{], }\AttributeTok{pch=}\DecValTok{4}\NormalTok{, }\AttributeTok{lwd=}\DecValTok{7}\NormalTok{)}
\end{Highlighting}
\end{Shaded}

\includegraphics{AS5R_files/figure-latex/unnamed-chunk-6-1.pdf}

\begin{Shaded}
\begin{Highlighting}[]
\FunctionTok{coefficients}\NormalTok{(mod.full, }\AttributeTok{id=}\DecValTok{3}\NormalTok{)}
\end{Highlighting}
\end{Shaded}

\begin{verbatim}
##              (Intercept) poly(x, 10, raw = TRUE)1 poly(x, 10, raw = TRUE)2 
##               2.06150718               0.97528027               2.87620901 
## poly(x, 10, raw = TRUE)3 
##              -0.08236142
\end{verbatim}

\begin{enumerate}
\def\labelenumi{(\alph{enumi})}
\setcounter{enumi}{3}
\tightlist
\item
  Repeat (c), using forward stepwise selection and also using backwards
  stepwise selection. How does your answer compare to the results in
  (c)?
\end{enumerate}

\begin{Shaded}
\begin{Highlighting}[]
\NormalTok{mod.fwd }\OtherTok{\textless{}{-}}  \FunctionTok{regsubsets}\NormalTok{(y }\SpecialCharTok{\textasciitilde{}} \FunctionTok{poly}\NormalTok{(x, }\DecValTok{10}\NormalTok{, }\AttributeTok{raw=}\ConstantTok{TRUE}\NormalTok{), }
                       \AttributeTok{data =}\NormalTok{ data.full, }\AttributeTok{nvmax =} \DecValTok{10}\NormalTok{, }
                       \AttributeTok{method=}\StringTok{"forward"}\NormalTok{)}

\CommentTok{\# backward}
\NormalTok{mod.bwd }\OtherTok{\textless{}{-}}  \FunctionTok{regsubsets}\NormalTok{(y }\SpecialCharTok{\textasciitilde{}} \FunctionTok{poly}\NormalTok{(x, }\DecValTok{10}\NormalTok{, }\AttributeTok{raw=}\ConstantTok{TRUE}\NormalTok{), }
                       \AttributeTok{data =}\NormalTok{ data.full, }\AttributeTok{nvmax=}\DecValTok{10}\NormalTok{, }
                       \AttributeTok{method=}\StringTok{"backward"}\NormalTok{)}

\NormalTok{fwd.summary }\OtherTok{\textless{}{-}}  \FunctionTok{summary}\NormalTok{(mod.fwd)}
\NormalTok{bwd.summary }\OtherTok{\textless{}{-}}  \FunctionTok{summary}\NormalTok{(mod.bwd)}

\FunctionTok{which.min}\NormalTok{(fwd.summary}\SpecialCharTok{$}\NormalTok{cp)}
\end{Highlighting}
\end{Shaded}

\begin{verbatim}
## [1] 4
\end{verbatim}

\begin{Shaded}
\begin{Highlighting}[]
\FunctionTok{which.min}\NormalTok{(bwd.summary}\SpecialCharTok{$}\NormalTok{cp)}
\end{Highlighting}
\end{Shaded}

\begin{verbatim}
## [1] 4
\end{verbatim}

\begin{Shaded}
\begin{Highlighting}[]
\FunctionTok{which.min}\NormalTok{(fwd.summary}\SpecialCharTok{$}\NormalTok{bic)}
\end{Highlighting}
\end{Shaded}

\begin{verbatim}
## [1] 2
\end{verbatim}

\begin{Shaded}
\begin{Highlighting}[]
\FunctionTok{which.min}\NormalTok{(bwd.summary}\SpecialCharTok{$}\NormalTok{bic)}
\end{Highlighting}
\end{Shaded}

\begin{verbatim}
## [1] 2
\end{verbatim}

\begin{Shaded}
\begin{Highlighting}[]
\FunctionTok{which.max}\NormalTok{(fwd.summary}\SpecialCharTok{$}\NormalTok{adjr2)}
\end{Highlighting}
\end{Shaded}

\begin{verbatim}
## [1] 4
\end{verbatim}

\begin{Shaded}
\begin{Highlighting}[]
\FunctionTok{which.max}\NormalTok{(bwd.summary}\SpecialCharTok{$}\NormalTok{adjr2)}
\end{Highlighting}
\end{Shaded}

\begin{verbatim}
## [1] 4
\end{verbatim}

\begin{Shaded}
\begin{Highlighting}[]
\FunctionTok{par}\NormalTok{(}\AttributeTok{mfrow=}\FunctionTok{c}\NormalTok{(}\DecValTok{3}\NormalTok{, }\DecValTok{2}\NormalTok{))}

\FunctionTok{plot}\NormalTok{(fwd.summary}\SpecialCharTok{$}\NormalTok{cp, }\AttributeTok{xlab=}\StringTok{"Subset Size"}\NormalTok{, }\AttributeTok{ylab=}\StringTok{"Forward Cp"}\NormalTok{, }\AttributeTok{pch=}\DecValTok{20}\NormalTok{, }\AttributeTok{type=}\StringTok{"l"}\NormalTok{)}
\FunctionTok{points}\NormalTok{(}\DecValTok{3}\NormalTok{, fwd.summary}\SpecialCharTok{$}\NormalTok{cp[}\DecValTok{3}\NormalTok{], }\AttributeTok{pch=}\DecValTok{4}\NormalTok{,  }\AttributeTok{lwd=}\DecValTok{7}\NormalTok{)}
\FunctionTok{plot}\NormalTok{(bwd.summary}\SpecialCharTok{$}\NormalTok{cp, }\AttributeTok{xlab=}\StringTok{"Subset Size"}\NormalTok{, }\AttributeTok{ylab=}\StringTok{"Backward Cp"}\NormalTok{, }\AttributeTok{pch=}\DecValTok{20}\NormalTok{, }\AttributeTok{type=}\StringTok{"l"}\NormalTok{)}
\FunctionTok{points}\NormalTok{(}\DecValTok{3}\NormalTok{, bwd.summary}\SpecialCharTok{$}\NormalTok{cp[}\DecValTok{3}\NormalTok{], }\AttributeTok{pch=}\DecValTok{4}\NormalTok{,  }\AttributeTok{lwd=}\DecValTok{7}\NormalTok{)}
\FunctionTok{plot}\NormalTok{(fwd.summary}\SpecialCharTok{$}\NormalTok{bic, }\AttributeTok{xlab=}\StringTok{"Subset Size"}\NormalTok{, }\AttributeTok{ylab=}\StringTok{"Forward BIC"}\NormalTok{, }\AttributeTok{pch=}\DecValTok{20}\NormalTok{, }\AttributeTok{type=}\StringTok{"l"}\NormalTok{)}
\FunctionTok{points}\NormalTok{(}\DecValTok{3}\NormalTok{, fwd.summary}\SpecialCharTok{$}\NormalTok{bic[}\DecValTok{3}\NormalTok{], }\AttributeTok{pch=}\DecValTok{4}\NormalTok{,  }\AttributeTok{lwd=}\DecValTok{7}\NormalTok{)}
\FunctionTok{plot}\NormalTok{(bwd.summary}\SpecialCharTok{$}\NormalTok{bic, }\AttributeTok{xlab=}\StringTok{"Subset Size"}\NormalTok{, }\AttributeTok{ylab=}\StringTok{"Backward BIC"}\NormalTok{, }\AttributeTok{pch=}\DecValTok{20}\NormalTok{, }\AttributeTok{type=}\StringTok{"l"}\NormalTok{)}
\FunctionTok{points}\NormalTok{(}\DecValTok{3}\NormalTok{, bwd.summary}\SpecialCharTok{$}\NormalTok{bic[}\DecValTok{3}\NormalTok{], }\AttributeTok{pch=}\DecValTok{4}\NormalTok{,  }\AttributeTok{lwd=}\DecValTok{7}\NormalTok{)}
\FunctionTok{plot}\NormalTok{(fwd.summary}\SpecialCharTok{$}\NormalTok{adjr2, }\AttributeTok{xlab=}\StringTok{"Subset Size"}\NormalTok{, }\AttributeTok{ylab=}\StringTok{"Forward Adjusted R2"}\NormalTok{, }\AttributeTok{pch=}\DecValTok{20}\NormalTok{, }\AttributeTok{type=}\StringTok{"l"}\NormalTok{)}
\FunctionTok{points}\NormalTok{(}\DecValTok{3}\NormalTok{, fwd.summary}\SpecialCharTok{$}\NormalTok{adjr2[}\DecValTok{3}\NormalTok{], }\AttributeTok{pch=}\DecValTok{4}\NormalTok{,  }\AttributeTok{lwd=}\DecValTok{7}\NormalTok{)}
\FunctionTok{plot}\NormalTok{(bwd.summary}\SpecialCharTok{$}\NormalTok{adjr2, }\AttributeTok{xlab=}\StringTok{"Subset Size"}\NormalTok{, }\AttributeTok{ylab=}\StringTok{"Backward Adjusted R2"}\NormalTok{, }\AttributeTok{pch=}\DecValTok{20}\NormalTok{, }\AttributeTok{type=}\StringTok{"l"}\NormalTok{)}
\FunctionTok{points}\NormalTok{(}\DecValTok{4}\NormalTok{, bwd.summary}\SpecialCharTok{$}\NormalTok{adjr2[}\DecValTok{4}\NormalTok{], }\AttributeTok{pch=}\DecValTok{4}\NormalTok{,  }\AttributeTok{lwd=}\DecValTok{7}\NormalTok{)}
\end{Highlighting}
\end{Shaded}

\includegraphics{AS5R_files/figure-latex/unnamed-chunk-9-1.pdf}

\begin{Shaded}
\begin{Highlighting}[]
\FunctionTok{coefficients}\NormalTok{(mod.fwd, }\AttributeTok{id=}\DecValTok{3}\NormalTok{)}
\end{Highlighting}
\end{Shaded}

\begin{verbatim}
##              (Intercept) poly(x, 10, raw = TRUE)1 poly(x, 10, raw = TRUE)2 
##               2.06150718               0.97528027               2.87620901 
## poly(x, 10, raw = TRUE)3 
##              -0.08236142
\end{verbatim}

\begin{Shaded}
\begin{Highlighting}[]
\FunctionTok{coefficients}\NormalTok{(mod.bwd, }\AttributeTok{id=}\DecValTok{3}\NormalTok{)}
\end{Highlighting}
\end{Shaded}

\begin{verbatim}
##              (Intercept) poly(x, 10, raw = TRUE)1 poly(x, 10, raw = TRUE)2 
##               2.06150718               0.97528027               2.87620901 
## poly(x, 10, raw = TRUE)3 
##              -0.08236142
\end{verbatim}

\begin{Shaded}
\begin{Highlighting}[]
\FunctionTok{coefficients}\NormalTok{(mod.fwd, }\AttributeTok{id=}\DecValTok{4}\NormalTok{)}
\end{Highlighting}
\end{Shaded}

\begin{verbatim}
##              (Intercept) poly(x, 10, raw = TRUE)1 poly(x, 10, raw = TRUE)2 
##               2.07200775               1.38745596               2.84575641 
## poly(x, 10, raw = TRUE)3 poly(x, 10, raw = TRUE)5 
##              -0.54202574               0.08072292
\end{verbatim}

\end{document}
